\documentclass{article}
\usepackage{graphicx}
\usepackage{changes}
\usepackage{lipsum}% <- For dummy text
\usepackage{subcaption}
\usepackage{cite}
\usepackage{natbib}
\bibliographystyle{plainnat}
%\bibliographystyle{plos2015}
%\usepackage{plainnat}
%\usepackage[doi=false]{biblatex}
\usepackage{setspace}
\usepackage{tablefootnote}
\usepackage{amsmath}% http://ctan.org/pkg/amsmath
\usepackage{kbordermatrix}% http://www.hss.caltech.edu/~kcb/TeX/kbordermatrix.sty
\usepackage[export]{adjustbox} %http://tex.stackexchange.com/questions/91566/syntax-similar-to-centering-for-right-and-left
\usepackage{authblk} %https://www.overleaf.com/help/75-how-do-i-add-additional-author-names-and-affiliations-to-my-paper#.VZ_wFvlVhBc

\captionsetup{compatibility=false}
\definechangesauthor[name={Per cusse}, color=orange]{per}

%\setremarkmarkup{(#2)}


\newcommand{\beginsupplement}{%
      \setcounter{table}{0}
      \renewcommand{\thetable}{S\arabic{table}}%
      \setcounter{figure}{0}
      \renewcommand{\thefigure}{S\arabic{figure}}%
   }
 
\begin{document}
\section{Introduction}
\subsection{Transmission Model}
The epidemiological and serological dynamics were simulated based on a disease transmission model with a serological response component. The serological model is an extension of the susceptible infected recovered (SIR) model to multiple levels, where each level represented a different antibody titre. 
Once an individual became infected, viruses could transmit to other susceptible individuals during the infectious periods. The transmission susceptibility for the titre given a contact was obtained based on the differential susceptibility. A higher antibody level would reduce the susceptibility of individuals. Once the individual got infected, the antibody was boosted to a higher level according to a truncated Poisson distribution. After recovery, the infected individual became  fully protected. At the same time, the antibody titres would be boosted to an elevated level.

Disease dynamics are described by the following equations while age mixing effects are considered. The model has not taken into account the demographics since the duration of the outbreak we considered here was less than one year. The birth and death rates would not produced significant different outcomes.

\begin{equation}
 \label{simple_equation}
 %\alpha = \sqrt{ \beta }
 \frac{dS{i}(a)}{dt}=-S_{i}\cdot\rho(i)\cdot\lambda(a)+\omega\cdot R_{i}(a)
\end{equation}

\begin{equation}
%\begin{split}%
 \frac{dI{i}(a)}{dt}=S_{i}\cdot\rho(i)\cdot\lambda(a)+\frac{1}{T_{g}}\sum_{j=i}^{i_{max}} I_{i}(a)\cdot g_{ij}
%\end{split}%
\end{equation}

\begin{equation}
 \frac{dR{i}(a)}{dt}=\frac{1}{T_{g}} \sum_{j=i}^{i_{max}} I_{i}(a)\cdot g_{ij}
\end{equation}

\clearpage
\begin{table}[ht]
\begin{minipage}{\textwidth}      
% Table generated by Excel2LaTeX from sheet 'Table1'
\centering % used for centering table
\caption{The parameters estimation from the titre model and the threshold model using MCMC. The miminum ESS is above 100.}
% title of Table
\centering % used for centering table
\begin{tabular}{rrrr}

\hline\hline \\%inserts double horizontal lines
Parameters* &          Descriptions &       Values \\ \\
\hline %inserts double horizontal lines
   test
    $K_{S}$ &	   association rate constant for HA binding to sialic acid receptors	  \\ \\
    
    $K_{B}$ &      association rate constant of antibody-antigen interaction \\ \\
  
    $K_{N}$ &      association binding constant for NA binding to HA-SA complex \\ \\

    $K_{\alpha}$ & the rate of virus entry into the cell through endocytosis    \\ \\

    $K_{\beta}$ &  the rate of antibody-mediated viral neutralization  \\ \\

    $K_{\gamma}$ & the rate of HA-SA complex cleavage by NA &     \\ \\
  
    $K_{d}$ &      the degradation rate of bound viruses upon viral release  \\ \\

\hline
\end{tabular}
\end{minipage}
\end{table}


where $a$ represents the age group \{1,2,3,4\} for each individual. $\rho(i)$ is the disease susceptibility for susceptible individuals in the presence of titres level , $\lambda$ is force of infection, $\omega$ is waning rate for CTL immunity. $T_{g}$ is the duration of infection 3.3 days. $g_{ij}$ is the probability of immune boosting from titres $i$ to $j$ following a Poisson distribution. The force of infection on member of age class $a$ who are completely naive within the population is 
\begin{equation}
 \lambda(a)=\beta\sum_{b=1}^{a_{max}}{\Big\{m_{ab}\sum_{i=0}^{i_{max}}\cdot I_{i}(b)\Big\}}
\end{equation}

where $\beta$ is the basic unit of transmission rate, $m_{ab}$ is the contact rate from the age class $b$ to $a$. We stratified sera samples into age groups, i.e. children and adolescent (2-19 y/o; for convenience, we defined this group as children throughout the study), young adults (20-39 y/o), mid-age adults (40-64 y/o) and elders ($\geq$= 65 y/o). Contact mixing matrix of the four age groups was calculated from a community study in Hongkong.

Initial seed $I_{0}$ in four different age groups are \{0.176 0.297 0.394 0.133\}. The sum of the initial seed is 1. Assuming they are are infected from naive titre group.

\subsection{Differential susceptibility}
The susceptibility $\rho$ is defined as $1-\phi$ where $\phi$ is the proportion of individuals that are protected from infection given a titre level. The protection is modelled in a two parameters logistic function.

\begin{equation}
 \phi(i) = \frac{1}{1+e^{I_{\beta}({i}-I_{\alpha})}}
\end{equation}
where $I_{\alpha}$ is defined as the antibody titre TP50, which is the titre, at which $\phi$ will drop $50\%$ from the maximum value (Figure S3A). $I_{\alpha}$ will be present as $\mathit{TP50}$ in our study. $I_{\beta}$ determines the shape of the curve.

The susceptibility $\rho(i)$ is \{
    1.0000
    0.9853
    0.8911
    0.5000
    0.1089
    0.0147
    0.0018
    0.0002
    0.0000
    0.0000  \}

\end{document}
